\documentclass[]{article}
\usepackage{lmodern}
\usepackage{amssymb,amsmath}
\usepackage{ifxetex,ifluatex}
\usepackage{fixltx2e} % provides \textsubscript
\ifnum 0\ifxetex 1\fi\ifluatex 1\fi=0 % if pdftex
  \usepackage[T1]{fontenc}
  \usepackage[utf8]{inputenc}
\else % if luatex or xelatex
  \ifxetex
    \usepackage{mathspec}
  \else
    \usepackage{fontspec}
  \fi
  \defaultfontfeatures{Ligatures=TeX,Scale=MatchLowercase}
\fi
% use upquote if available, for straight quotes in verbatim environments
\IfFileExists{upquote.sty}{\usepackage{upquote}}{}
% use microtype if available
\IfFileExists{microtype.sty}{%
\usepackage{microtype}
\UseMicrotypeSet[protrusion]{basicmath} % disable protrusion for tt fonts
}{}
\usepackage{hyperref}
\hypersetup{unicode=true,
            pdftitle={Information Modeling},
            pdfauthor={University of Illinois School of Information Sciences},
            pdfborder={0 0 0},
            breaklinks=true}
\urlstyle{same}  % don't use monospace font for urls
\IfFileExists{parskip.sty}{%
\usepackage{parskip}
}{% else
\setlength{\parindent}{0pt}
\setlength{\parskip}{6pt plus 2pt minus 1pt}
}
\setlength{\emergencystretch}{3em}  % prevent overfull lines
\providecommand{\tightlist}{%
  \setlength{\itemsep}{0pt}\setlength{\parskip}{0pt}}
\setcounter{secnumdepth}{0}
% Redefines (sub)paragraphs to behave more like sections
\ifx\paragraph\undefined\else
\let\oldparagraph\paragraph
\renewcommand{\paragraph}[1]{\oldparagraph{#1}\mbox{}}
\fi
\ifx\subparagraph\undefined\else
\let\oldsubparagraph\subparagraph
\renewcommand{\subparagraph}[1]{\oldsubparagraph{#1}\mbox{}}
\fi

\title{Information Modeling}
\author{University of Illinois School of Information Sciences}
\date{Last updated Wed, May 31, 2017 9:10:04 PM}

\begin{document}
\maketitle

LIS561-LE\\
Spring 2017\\
Mondays, 5:30-7:30 PM, online\\
4 GR hours

Instructor: David Dubin\\
Email: ddubin@illinois.edu\\
Office: LIS 330\\
Office Hour: Tuesdays, 3-5pm and by appointment\\
Phone: (217) 244-3275

\section{Course Description}\label{course-description}

An introduction to the foundations of information modeling methods used
in current information management applications. The specific methods
considered include relational database design, conceptual modeling, and
ontologies. The basic concepts underlying these methods are sets,
relations, entities, and logics. Applications considered include
relational database design and RDF/OWL semantic web languages. Set
theory and logic are emphasized as the foundational frameworks for
information modeling in general, and for contemporary web-based
information management and delivery systems (including semantic web
technologies) in particular.

\subsection{Pre- and Co-requisites}\label{pre--and-co-requisites}

None.

\section{Course Overview}\label{course-overview}

Two sorts of students are anticipated and the course objectives are
similar but slightly different for each group. In neither case is prior
relevant knowledge assumed.

\begin{itemize}
\tightlist
\item
  LIS561 prepares students anticipating generalist responsibilities (as
  directors, managers, general staff, etc.) to be effective leaders in
  making decisions about the design, development, and evaluation of
  information systems, services, and policies, helping their
  organizations and communities deal with all aspects of the difficult
  technology challenges ahead.
\item
  LIS561 prepares students anticipating careers as technology
  specialists to efficiently acquire and maintain superior information
  modeling skills throughout their careers and to play leadership roles
  in the design, development, and evaluation of information systems,
  services, and policies.
\end{itemize}

Consistent with the iSchool goal of producing leaders and not just
competent professionals we focus on developing a deep understanding that
will have long-term benefits and prepare students to engage the hardest
problems facing organizations and society.

Of course LIS561 alone cannot fully realize these objectives; it makes a
partial contribution, focusing on the principles and concepts of
information modeling. A partial contribution, but a necessary one: the
connection between a deep understanding of information modeling concepts
and the challenging information management problems facing us today is
profound.

\subsection{Strategy}\label{strategy}

The course examines the major modeling approaches currently in use in
information management: relational modeling, conceptual modeling, and
ontologies, focusing on underlying concepts and principles. The course
is thus simultaneously a foundations course and a survey course.

\subsection{Learning Objectives}\label{learning-objectives}

\begin{enumerate}
\def\labelenumi{\arabic{enumi}.}
\tightlist
\item
  Develop fluency in reading and understanding formal definitions.
\item
  Understand the role of abstraction in making systems design choices.
\item
  Contrast deep vs.~superficial differences in modeling languages.
\item
  Recognize practical implications of trading expressive power for
  tractability.
\item
  Appreciate the fundamental role of a very small set of inter-related
  concepts.
\end{enumerate}

\section{Course Materials}\label{course-materials}

All required readings for this class are available online. They are
listed in the references section at the end of this syllabus.

\section{About Dave Dubin}\label{about-dave-dubin}

David Dubin is a Research Associate Professor at the School of
Information Sciences. His research explores the foundations of
information representation and description as well as issues of
expression and encoding in documents and digital information resources.

\section{Library Resources}\label{library-resources}

http://www.library.illinois.edu/lis/\\
lislib@library.illinois.edu\\
Phone: (217) 300-8439

\section{Writing and Bibliographic Style
Resources}\label{writing-and-bibliographic-style-resources}

The iSchool has a Writing Resources Moodle site
\url{https://courses.ischool.illinois.edu/course/view.php?id=1705} and
iSchool writing coaches also offer free consultations. We highly
recommend this!

The campus-wide Writers Workshop also provides free consultations. For
more information see \url{http://www.cws.illinois.edu/workshop/}

\section{Academic Integrity}\label{academic-integrity}

Please review and reflect on the academic integrity policy of the
University of Illinois,
\url{http://admin.illinois.edu/policy/code/article1_part4_1-401.html} to
which we subscribe. By turning in materials for review, you certify that
all work presented is your own and has been done by you independently,
or as a member of a designated group for group assignments. If, in the
course of your writing, you use the words or ideas of another writer,
proper acknowledgment must be given (using APA, Chicago, or MLA style).
Not to do so is to commit plagiarism, a form of academic dishonesty. If
you are not absolutely clear on what constitutes plagiarism and how to
cite sources appropriately, now is the time to learn. Please ask me!
Please be aware that the consequences for plagiarism or other forms of
academic dishonesty will be severe. Students who violate university
standards of academic integrity are subject to disciplinary action,
including a reduced grade, failure in the course, and suspension or
dismissal from the University.

\section{Statement of Inclusion}\label{statement-of-inclusion}

\href{http://www.inclusiveillinois.illinois.edu/supporting_docs/Inclusive\%20Illinois\%20Diversity\%20Statement.pdf}{Inclusive
Illinois Committee Diversity Statement}

As the state's premier public university, the University of Illinois at
Urbana-Champaign's core mission is to serve the interests of the diverse
people of the state of Illinois and beyond. The institution thus values
inclusion and a pluralistic learning and research environment, one which
we respect the varied perspectives and lived experiences of a diverse
community and global workforce. We support diversity of worldviews,
histories, and cultural knowledge across a range of social groups
including race, ethnicity, gender identity, sexual orientation,
abilities, economic class, religion, and their intersections.

\section{Accessibility Statement}\label{accessibility-statement}

To obtain accessibility-related academic adjustments and/or auxiliary
aids, students with disabilities must contact the course instructor and
the \href{http://disability.illinois.edu/}{Disability Resources and
Educational Services} (DRES) as soon as possible. To contact DRES you
may visit 1207 S. Oak St., Champaign, call (217) 333-4603 (V/TTY), or
e-mail a message to disability@illinois.edu.

\subsection{Emergency response: Run, Hide,
Fight}\label{emergency-response-run-hide-fight}

Emergencies can happen anywhere and at any time. It is important that we
take a minute to prepare for a situation in which our safety or even our
lives could depend on our ability to react quickly. When we're faced
with any kind of emergency -- like fire, severe weather or if someone is
trying to hurt you -- The
\href{http://police.illinois.edu/safe}{University of Illinois Police
Department} recommends three options:
\href{http://police.illinois.edu/dpsapp/wp-content/uploads/2016/08/syllabus-attachment.pdf}{Run,
hide or fight}.

\section{Assignments and Evaluation}\label{assignments-and-evaluation}

All assignments are required for all students. All work must be
completed in order to pass this class. Late or incomplete assignments
will not be given full credit unless the student has contacted the
instructor prior to the due date of the assignment (or in the case of
emergencies, as soon as practicable).

\textbf{Assignments, Exercises \& Grade Distribution:}

\begin{itemize}
\tightlist
\item
  Eleven graded assignments (due 1 hour before class meeting in the week
  they are due): 5 points each (55 points total).
\item
  Eleven ungraded exercises (due 25 hours before class meeting in the
  week they are due): 2 points each for completion (22 points total).
\item
  Four reading responses: 5 points each (20 points total).
\item
  Attendance and participation in class and on forums: 3 points.
\end{itemize}

\textbf{Grading Scale:}

94-100 = A\\
90-93 = A-\\
87-89 = B+\\
83-86 = B\\
80-82 = B-\\
77-79 = C+\\
73-76 = C\\
70-72 = C-\\
67-69 = D+\\
63-66 = D\\
60-62 = D-\\
59 and below = F

\textbf{Reading Responses}

Up to twenty points are available for four critical questions posted to
the class discussion forum and identified in the subject line as a
\emph{reading response} together with a topic description. Critical
questions are carefully worded descriptions of an obstacle to
understanding and applying concepts and methods covered in the assigned
readings. These need to be more than a simple request for clarifying a
term or idea. Each question should be one to three paragraphs in length,
and should include:

\begin{itemize}
\tightlist
\item
  A one sentence summary of the question (citing the location of the
  anomalous passage in the reading);
\item
  A longer, lucid explanation of the question and the obstacle that it
  identifies;
\item
  A justification for why the question is important;
\item
  Some remarks on the context in which the question arose: how did you
  recognize that it was a problem?
\end{itemize}

\section{Topic Schedule}\label{topic-schedule}

\subsubsection{Week 2: January 23: Models and
Domains}\label{week-2-january-23-models-and-domains}

\subsubsection{Week 3: January 30: Propositional
Logic}\label{week-3-january-30-propositional-logic}

\textbf{Required Readings:} Benthem et al. 2014a

\textbf{Due:} \textbf{Graded Assignment 1}, SVG diagram exercise

\textbf{Due:} \textbf{Ungraded Exercise 1}, Propositional logic exercise

\subsubsection{Week 4: February 06: Predicate
Logic}\label{week-4-february-06-predicate-logic}

\textbf{Required Readings:} Benthem et al. 2014b

\textbf{Due:} \textbf{Ungraded Exercise 2}, Predicate logic exercise 1

\textbf{Due:} \textbf{Graded Assignment 2}, Propositional logic
assignment

\subsubsection{Week 5: February 13: Predicate
Logic}\label{week-5-february-13-predicate-logic}

\textbf{Required Readings:} Benthem et al. 2014b

\textbf{Due:} \textbf{Ungraded Exercise 3}, Predicate logic exercise 2

\textbf{Due:} \textbf{Graded Assignment 3}, Predicate logic assignment

\subsubsection{Week 6: February 20: Sets, relations, and
functions}\label{week-6-february-20-sets-relations-and-functions}

\textbf{Required Readings:} Partee 2006

\textbf{Due:} \textbf{Ungraded Exercise 4}, Set theory exercise

\subsubsection{Week 7: February 27: UML and relational
modeling}\label{week-7-february-27-uml-and-relational-modeling}

\textbf{Required Readings:} Seidl et al. 2015; Teorey et al. 1986

\textbf{Due:} \textbf{Ungraded Exercise 5}, UML class diagram exercise

\textbf{Due:} \textbf{Graded Assignment 4}, Set theory assignment

\subsubsection{Week 8: March 06: Normal forms and
normalization}\label{week-8-march-06-normal-forms-and-normalization}

\textbf{Required Readings:} Kent 1983

\textbf{Due:} \textbf{Ungraded Exercise 6}, Relational modeling exercise

\textbf{Due:} \textbf{Graded Assignment 5}, UML class diagram assignment

\subsubsection{Week 9: March 13: Syntax and
Grammar}\label{week-9-march-13-syntax-and-grammar}

\textbf{Required Readings:} Rosen 1988

\textbf{Due:} \textbf{Ungraded Exercise 7}, Formal grammar exercise

\textbf{Due:} \textbf{Graded Assignment 6}, Relational modeling
assignment

\subsubsection{Week 10: Spring Break: March
20:}\label{week-10-spring-break-march-20}

\subsubsection{Week 11: March 27: Semantics and
Interpretation}\label{week-11-march-27-semantics-and-interpretation}

\textbf{Required Readings:} Bach 1989

\textbf{Due:} \textbf{Graded Assignment 7}, Formal grammar assignment

\textbf{Due:} \textbf{Ungraded Exercise 8}, Formal semantics exercise

\subsubsection{Week 12: April 03: The RDF model and
language}\label{week-12-april-03-the-rdf-model-and-language}

\textbf{Required Readings:} Manola et al. 2014

\textbf{Due:} \textbf{Ungraded Exercise 9}, RDF description exercise

\textbf{Due:} \textbf{Graded Assignment 8}, Formal semantics assignment

\subsubsection{Week 13: April 10: Description
Logics}\label{week-13-april-10-description-logics}

\textbf{Required Readings:} Krötzsch et al. 2012

\textbf{Due:} \textbf{Graded Assignment 9}, RDF description assignment

\textbf{Due:} \textbf{Ungraded Exercise 10}, Description logic exercise

\subsubsection{Week 14: April 17:
Ontologies}\label{week-14-april-17-ontologies}

\textbf{Required Readings:} Hitzler et al. 2012; Manola et al. 2014

\textbf{Due:} \textbf{Ungraded Exercise 11}, OWL ontology exercise

\textbf{Due:} \textbf{Graded Assignment 10}, Description logic
assignment

\subsubsection{Week 15: April 24: Looking
Ahead}\label{week-15-april-24-looking-ahead}

\textbf{Due:} \textbf{Graded Assignment 11}, OWL ontology assignment

\subsubsection{Week 16: May 1: Wrapup and
Evaluation}\label{week-16-may-1-wrapup-and-evaluation}

\section*{Readings}\label{readings}
\addcontentsline{toc}{section}{Readings}

\hypertarget{refs}{}
\hypertarget{ref-bachux5fbackgroundux5f1989}{}
Bach, E. 1989. ``Background and Beginning, Worlds Enough and Time.'' In
\emph{Informal Lectures on Formal Semantics}. Albany, NY, 1--32.
\url{https://uofi.box.com/s/lfqsrzjkhzdzml9d2g5w0ndtyvn0ndom}.

\hypertarget{ref-vanux5fbenthemux5fchapterux5f2014}{}
Benthem, J van, Ditmarsch, H van, Eijck, J van, and Jaspars, J. 2014a.
``Chapter 2: Propositional Logic''. In \emph{Logic in Action}.
Amsterdam, NL, 2.1--2.37.
\url{http://www.logicinaction.org/docs/ch2.pdf}.

\hypertarget{ref-vanux5fbenthemux5fchapterux5f2014-1}{}
Benthem, J van, Ditmarsch, H van, Eijck, J van, and Jaspars, J. 2014b.
``Chapter 4: The World According to Predicate Logic''. In \emph{Logic in
Action}. Amsterdam, NL, 4.1--4.53.
\url{http://www.logicinaction.org/docs/ch4.pdf}.

\hypertarget{ref-hitzlerux5fowlux5f2012}{}
Hitzler, P, Krötzsch, M, Parsia, B, Patel-Schneider, P F, and Rudolph,
S. 2012. ``OWL 2 Web Ontology Language Primer''.
\url{http://www.w3.org/TR/owl2-primer/}.

\hypertarget{ref-kentux5fsimpleux5f1983}{}
Kent, W. 1983. ``A Simple Guide to Five Normal Forms in Relational
Database Theory''. \emph{Commun. ACM} 26.2, 120--125.
\url{http://doi.acm.org.proxy2.library.illinois.edu/10.1145/358024.358054}.

\hypertarget{ref-krotzschux5fdescriptionux5f2012}{}
Krötzsch, M, Simancík, F, and Horrocks, I. 2012. ``A Description Logic
Primer''. \emph{arXiv preprint arXiv:1201.4089}.
\url{http://arxiv.org/abs/1201.4089}.

\hypertarget{ref-manolaux5frdfux5f2014}{}
Manola, F, Miller, E, and McBride, B. 2014. ``RDF 1.1 Primer''.,
Cambridge, MA.
\url{https://www.w3.org/TR/2014/NOTE-rdf11-primer-20140624/}.

\hypertarget{ref-parteeux5fbasicux5f2006}{}
Partee, B H. 2006. ``Basic Concepts of Set Theory, Functions and
Relations''.
\url{http://people.umass.edu/partee/NZ_2006/Set\%20Theory\%20Basics.pdf}.

\hypertarget{ref-rosenux5flanguagesux5f1988}{}
Rosen, K H. 1988. ``Languages and Grammars''. In \emph{Discrete
Mathematics and its Applications}. New York, 552--563.
\url{https://uofi.box.com/s/nomrry0e4cone88xvnciaf14gg93t68h}.

\hypertarget{ref-seidlux5fclassux5f2015}{}
Seidl, M, Scholz, M, Huemer, C, and Kappel, G. 2015. ``The Class
Diagram''. In \emph{UML @ Classroom: An Introduction to Object-Oriented
Modeling}. Eds. M. Seidl, M. Scholz, C. Huemer, and G. Kappel. Cham,
49--84. \url{http://dx.doi.org/10.1007/978-3-319-12742-2_4}.

\hypertarget{ref-Teoreyux5f1986}{}
Teorey, T J, Yang, D, and Fry, J P. 1986. ``A logical design methodology
for relational databases using the extended entity-relationship model''.
\emph{ACM Comput. Surv.} 18.2, 197--222.
\url{http://doi.acm.org/10.1145/7474.7475}.

\end{document}
